\documentclass{article}

\usepackage[utf8]{inputenc}
\usepackage[T1]{fontenc}
\usepackage[francais]{babel}
\usepackage[top=2cm, bottom=2cm, left=2cm, right=2cm]{geometry}
\usepackage{tabularx}
\usepackage{graphics}
\usepackage{hyperref}

\title{Compte Rendu\\TPE}
\author{Luc Chabassier \and Pablo Donato \and Nicolas Rasmont \and Lucas Saliou \and Arthur Sirech}

\begin{document}
\maketitle
\tableofcontents

\section{Analyse du besoin}
Ce produit a pour but de rendre service aux personnes désœuvrées, en leur fournissant un moyen de divertissement ne nécessitant pas de partenaire humain. Voir la bête à cornes page \pageref{bete_cornes}.
\begin{figure}
	\begin{center}
		\includegraphics{bete_corne.png}
	\end{center}
	\caption{Bête à cornes}
	\label{bete_cornes}
\end{figure}

\section{Cahier des charges}
Ce projet a pour objectif de créer une machine capable de fournir un adversaire de jeu lors de parties se rapprochant du jeu de fléchettes.

\subsection{Les règles du jeu}
Le principe du jeu est d'envoyer de petites billes magnétiques sur une cible métallique sur laquelle sont représentés des cercles concentriques. Chaque cercle apporte un certain nombre de points, augmentant au fur et à mesure que l'on se rapproche du centre. Chaque joueur envoie à tour de rôle une bille. Le nombre de billes pour chaque joueur est décidé en début de partie. À la fin, chaque joueur compte ses points : celui qui en a le plus a gagné.

\subsection{Le projet}
Notre projet consiste en un adversaire mécanique pour un jeu dont les règles sont présentées ci-dessus.

\subsubsection{Initialisation}
Le robot doit en début de partie être placé manuellement devant le centre de la cible : un laser pointeur est prévu à cet effet. La distance à la cible ne doit pas excéder 4m (distance non définitive). Une fois placé, on allume le robot avec un interrupteur équipé d'un indicateur sous forme de LED. Après quelques secondes, la LED doit se mettre à clignoter. À ce moment, vous pouvez vous y connecter à partir du programme android fourni avec le robot.

\subsubsection{Le programme}
Le programme communiquera avec le robot par Wifi. Il doit permettre de lancer une partie, choisir le niveau de difficulté ainsi que le nombre de billes par joueur, et indiquer au robot quand tirer.
% TODO le lien
\paragraph{Téléchargement} Le programme sera stocké sous forme d'un fichier apk sur le compte github du projet et téléchargeable à cette adresse. Si possible, il sera aussi disponible sur Google Play.

\subsubsection{Le robot}
\begin{figure}
	\begin{center}
		\includegraphics{pieuvre2.png}
	\end{center}
	\caption{Diagramme pieuvre}
	\label{pieuvre}
\end{figure}

Une fois placé (donc lors de l'allumage), le robot calcule sa distance à la cible à l'aide d'un module sonar. Durant une partie, en fonction du niveau de difficulté choisi par l'utilisateur, il va déterminer pour chaque bille une position aléatoire. Ceci fait, il va calculer l'angle dans lequel il doit se placer et va tirer. Le canon peut pivoter dans une fenêtre de 45\degre avec une précision d'un demi degré. Puis il reste dans sa position jusqu'à ce qu'on lui indique de tirer à nouveau ou que la partie soit terminée.
\paragraph{Matériel} Les calculs seront réalisé sur une carte \emph{Arduino Duemilanove}, la communication Wifi par le shield officiel d'arduino, le positionnement par un moteur pas à pas et le tir par un \emph{canon Gauss}.
\paragraph{} Les différentes fonctions de services sont présentées dans le diagramme pieuvre page \pageref{pieuvre}.

\subsubsection{L'alimentation}
L'alimentation sera assurée par deux piles de 9V.

\section{Répartition des tâches}
\begin{center}
	\begin{tabular}{|l|c|}
		\hline
		Luc Chabassier 	& Programmation de la partie Arduino du projet. \\
		\hline
		Pablo Donato	& Programmation de l'application Android. \\
		\hline
		Nicolas Rasmont	& Réalisation support et canon Gauss. \\
		\hline
		Lucas Saliou	& Réalisation support et canon Gauss. \\
		\hline
		Arthur Sirech	& Modélisation et simulation informatique. \\
		\hline
	\end{tabular}
\end{center}

\end{document}

